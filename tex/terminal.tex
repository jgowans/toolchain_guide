\chapter{The Terminal}
If you're familair with the terminal, skip this section.\\

When we run these programs which have just been discussed, we will use the terminal to run them and view their output. 
The terminal is known as command prompt in Windows and as many things including the command line, shell or bash or terminal in Linux.
The Linux terminal is much better and nicer to use than the Windows one. This is mainly because a lot of Linux development work is done in the terminal and as such it needs to be really good for productivity reasons. Most Windows users are not real programmers, so the command prompt which ships with Windows is a bit rubbish. \\

In light of this, if you've never used Linux before and are keen to give it a try, I'd strongly recommend it. For the sort of work we will be doing, it is easier and more enjoyable to use.

I'm not going to cover how to use the terminal in this guide as that has already been documented very well in other places.

For Linux users, a brief introduction to the terminal can be found here: \url{http://linuxcommand.org/index.php}. Additionally, there is an open source 

\section{The Linux Terminal}
Assuming you're in Ubuntu, a terminal is launched by pressing Ctrl+Shift+T. Alternatively, you can press your Super key and then type ``terminal'' and you should be presented with a launcher for it which you can click on.

Once in the terminal, you can print out your current working directory with the command \texttt{pwd}. You can change directory with the \texttt{cd} command. For example:
\begin{lstlisting}[language=bash]
  $ cd /home/$USER/Documents
\end{lstlisting} 
to change to the documents directory, or simply 
\begin{lstlisting}[language=bash]
  $ cd ./Documents
\end{lstlisting} 
assuming your working directory is already your home The terminal 

A terminal always operates in a specific directory. When you launch a terminal, the default directory is usually your home directory.
Whenever you try to run a program or modify a file from the terminal, it attempts to modify the file which is inside the working directory which the terminal is set to.
In order to see what working directory your terminal is in, you can use the command \emph{pwd} (print working directory).
In order to change the working directory to a different one, you can use the comma

\lstdefinelanguage
   [x64]{Assembler}     % add a "x64" dialect of Assemblers
   [x86masm]{Assembler} % based on the "x86masm" dialect
   % with these extra keywords:
   {morekeywords={CDQE,CQO,CMPSQ,CMPXCHG16B,JRCXZ,LODSQ,MOVSXD, %
                  POPFQ,PUSHFQ,SCASQ,STOSQ,IRETQ,RDTSCP,SWAPGS, %
                  rax,rdx,rcx,rbx,rsi,rdi,rsp,rbp, %
                  r8,r8d,r8w,r8b,r9,r9d,r9w,r9b}} % etc.

\lstset{language=[x64]Assembler}

\begin{lstlisting}
  cdqe 1, r8
  push 1
  add rsp, 4
  push 1
\end{lstlisting}

