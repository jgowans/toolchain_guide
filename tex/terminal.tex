\chapter{The Terminal}
If you're familiar with the terminal, skip this section.\\

When we run these programs which have just been discussed, we will use the terminal to run them and view their output. 
The terminal is known as command prompt in Windows and as many things including the command line, shell or bash or terminal in Linux.
The Linux terminal is much better and nicer to use than the Windows one. 
This is mainly because a lot of Linux development work is done in the terminal and as such it needs to be really good for productivity reasons. 

In light of this, if you've never used Linux before and are keen to give it a try, I'd strongly recommend it. For the sort of work we will be doing, it is easier and more enjoyable to use.

I'm not going to cover how to use the terminal in this guide as that has already been documented very well in other places.\\

For Linux users, a brief introduction to the terminal can be found here: \url{http://linuxcommand.org/}. Additionally, there is a free book covering the contents of this website called The Linux Command Line which has been uploaded to \verb+Resources\Further_Reading+. I advise you be familiar with at least the first two chapters, preferable the first four.\\

For Windows users, there is a short introduction to command prompt found at \url{http://dosprompt.info/}. You should read through and practice everything discussed in that web page.\\

Henceforth, you are assumed to have adequate knowledge of the terminal.

%\lstdefinelanguage
%   [x64]{Assembler}     % add a "x64" dialect of Assemblers
%   [x86masm]{Assembler} % based on the "x86masm" dialect
%   % with these extra keywords:
%   {morekeywords={CDQE,CQO,CMPSQ,CMPXCHG16B,JRCXZ,LODSQ,MOVSXD, %
%                  POPFQ,PUSHFQ,SCASQ,STOSQ,IRETQ,RDTSCP,SWAPGS, %
%                  rax,rdx,rcx,rbx,rsi,rdi,rsp,rbp, %
%                  r8,r8d,r8w,r8b,r9,r9d,r9w,r9b}} % etc.
%
%\lstset{language=[x64]Assembler}
%
%\begin{lstlisting}
%  cdqe 1, r8
%  push 1
%  add rsp, 4
%  push 1
%\end{lstlisting}

